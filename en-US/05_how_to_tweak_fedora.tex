\chapter*{How to Tweak Fedora Workstation?}
\section*{GNOME Shell Extensions}

\emph{GNOME Shell extensions} are an invaluable tool when it comes to customizing the system for your particular needs. There are hundreds of extensions that change individual elements of the user interface or add new ones. With these extensions, you can change the look and feel of various menus, icons, panels, indications, displays, window switching, and many other things.

You can explore them and install \emph{GNOME Shell extensions} in \emph{Software} (under Addons/Shell Extensions), or directly in your web browser by visiting \url{extensions.gnome.org}.

Just keep in mind that the extensions can have an impact on the stability of your desktop. They're also third-party software the \emph{Fedora Project} doesn't give any guarantees for. So be careful what extensions you enable, and if you experience any stability problems, disable them.

\section*{GNOME Tweaks}

Fedora also includes \emph{GNOME Tweaks}, a powerful utility that allows you to fine tune your desktop and change settings that are not available in the default configuration tool. With this utility, you can adjust the behavior of virtual workspaces, the behavior of the system during charging, application fonts, keyboard shortcuts, and much more. As a~matter of fact, you can even use \emph{GNOME Tweaks} to manage the \emph{GNOME Shell extensions} mentioned above.
\endinput
