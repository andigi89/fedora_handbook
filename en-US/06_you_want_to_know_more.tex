\section*{Where to Look for Further Information}

This handbook exists to help you get started with Fedora Workstation. But, what can you do if you encounter a problem that you can't solve quickly? A good rule of thumb is to remember that any problem you're encountering has probably already been solved by somebody else before you. Here is where to look for solutions:
\begin{itemize}
\item\url{http://ask.fedoraproject.org} -- the official Q\&A forum with a large user base and a long history of solved problems.

\item\url{http://fedoramagazine.org} -- articles from Fedora world, including announcements and information about new applications.
\end{itemize}

\subsection*{What to Do If You Encounter a Bug}

It is possible that you will encounter a bug. What then? Fedora uses a bug tracker called Bugzilla that is provided by Red~Hat and you can find it online at \url{http://bugzilla.redhat.com}. Keep in mind that bugs must be reported in English.

Fedora also comes with a tool called ABRT. When a problem occurs, ABRT detects it and allows you to report a bug by just agreeing to share the details with the Fedora Project.

Reporting a bug is a great way to contribute to Fedora and help us make it better.

\subsection*{Fedora Editions}

Unlike many other Linux operating systems, Fedora consists of three main editions. In addition to the \emph{Workstation} edition, which is described in this guide, there are also \emph{Server} and \emph{Atomic} editions:
\begin{itemize}
\item The \emph{Server} edition is designed for server uses and features applications like Cockpit which allows for easy remote management of servers through a web browser and the administration of server roles through \emph{Rolekit}.

\item The \emph{Atomic} edition is a minimal version of Fedora optimized for deploying containers. There are multiple Fedora Atomic images optimized for use in environments like \emph{OpenStack}, \emph{VirtualBox}, and others.
\end{itemize}

\subsection*{Fedora Spins}

Everything in this guide is about the default Fedora \emph{Workstation} edition, which uses \emph{GNOME~3} and \emph{GNOME Shell}. Because there are many other popular desktop environments, so called spins offer alternative software builds of Fedora with these environments. Do you prefer the \emph{KDE Plasma Desktop}? You'll find an installation image with it. \emph{Xfce}? \emph{LXDE}? We have prepared installation images for everything.

But we don't have to stop with Fedora spins. The ARM architecture is currently the most used mobile platform and is the platform of choice for development boards like Banana Pi, BeagleBone, or the Raspberry Pi. There are images built to run on these kinds of systems too in the form of server versions or minimal installation images.
\endinput
