\chapter*{Chcete vědět víc?}

\section*{Kam se podívat dál?}
Naše příručka má skromné ambice, ale co když narazíte na problém, který nejste schopni rychle vyřešit? Základní pravidlo zní: problém, který řešíte, už takřka jistě řešil někdo před vámi. Kam se tedy podívat?

\begin{itemize}
\item \url{fedora.cz} -- z~českých webů ve vztahu k~Fedoře zásadní stránky (a univerální rozcestník). Obsahuje množství článků, tipů a dalších informací.
\item \url{wiki.fedora.cz} -- příručka o~Fedoře, na které mnoho lidí odvedlo obrovské množství práce. Mnoho návodů, které hledáte, budou právě zde.
\item \url{forum.fedora.cz} -- fórum v~českém jazyce hodící se vždy, když nejste schopni problém vyřešit sami nebo když sami chcete nabídnout svou pomoc.
\item \url{fedoraforum.org} -- fórum v~anglickém jazyce s~rozsáhlou uživatelskou základou a dlouhou historií řešených problémů. Nyní už byste mohli být v~cíli.
\item \url{fedoramagazine.org} -- články o~dění kolem Fedory. Nové aplikace a oznámení.
\end{itemize}

\section*{Co když narazím na chybu?}
Je docela možné, že se dostanete do situace, kdy narazíte na chybu zcela novou. Co pak? Fedora používá bugzillu společnosti Red Hat, která je dostupná z~adresy \url{bugzilla.redhat.com}. (Nutno podotknout, že nezbytností při hlášení chyb je angličtina, alespoň na základní úrovni.) Není to ale jediná možnost, jak chybu ohlásit. Ve Fedoře je dostupný nástroj ABRT. Budeme-li citovat klasický snímek, pak daný nástroj vystihuje spojení \emph{\uv{každý přispívá podle svých možností}} a ABRT koná přesně v~tom duchu.

\section*{Další edice Fedory}
Zde se dostáváme k~věcem, ve kterých se Fedora odlišuje od obecného spektra linuxových operačních systémů. Fedora doznala své podoby ve třech hlavních edicích. Už dříve popsaná je edice \emph{Workstation}, dále pak tu máme edice \emph{Server} a \emph{Cloud}. Pojďme na ně. U~edice \emph{Server} lze vyzdvihnout aplikace jako Cockpit sloužící pro vzdálenou správu běžících serverů prostřednictvím webového prohlížeče nebo administraci serverových rolí přes Rolekit. U~edice \emph{Cloud} pak máme k~dispozici minimalistickou verzi Fedory, která umožňuje akorát nasazení kontejnerů. A~že existuje hned několik obrazů Fedory Cloud připravených pro nasazení v~prostředí \emph{OpenStack}, \emph{VirtualBox} a dalších.

\section*{Fedora a spiny}
Vše, co dosud padlo ve vztahu k~Fedora \emph{Workstation}, se týkalo výchozího cílení toho operačního systému, jehož desktopovým prostředím je \emph{GNOME~3} a jeho \emph{GNOME Shell}. Co když se ale chceme rozhlédnout dále? Ve Fedoře existují tzv. spiny, čili připravené instalační obrazy se specifickým cílením. Jste zvyklí na \emph{KDE Plasma}? Je tu \emph{KDE Plasma Desktop}. \emph{Xfce}? \emph{LXDE}? Pro vše tu jsou připravené obrazy. Ale ani u~Fedora spinů se nemusíme zastavit. V~dnešní době je velmi aktuální procesorová architektura ARM (nejčastější mobilní platforma, platforma pro různé vývojové desky à la Banana Pi, BeagleBone, nebo v~případě Fedora Remixu i Raspberry Pi). Stejně jako některé další linuxové operační systémy i Fedora má pro tuto architekturu připravené řešení ve formě jak serverové verze, tak minimálního instalačního obrazu pro obecné použití.
